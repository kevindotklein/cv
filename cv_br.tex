\documentclass[10pt, letterpaper]{article}

% Packages:
\usepackage[
    ignoreheadfoot, % set margins without considering header and footer
    top=2 cm, % seperation between body and page edge from the top
    bottom=2 cm, % seperation between body and page edge from the bottom
    left=2 cm, % seperation between body and page edge from the left
    right=2 cm, % seperation between body and page edge from the right
    footskip=1.0 cm, % seperation between body and footer
    % showframe % for debugging 
]{geometry} % for adjusting page geometry
\usepackage{titlesec} % for customizing section titles
\usepackage{tabularx} % for making tables with fixed width columns
\usepackage{array} % tabularx requires this
\usepackage[dvipsnames]{xcolor} % for coloring text
\definecolor{primaryColor}{RGB}{0, 79, 144} % define primary color
\usepackage{enumitem} % for customizing lists
\usepackage{fontawesome5} % for using icons
\usepackage{amsmath} % for math
\usepackage[
    pdftitle={cv-br},
    pdfauthor={Kevin Klein},
    pdfcreator={LaTeX},
    colorlinks=true,
    urlcolor=primaryColor
]{hyperref} % for links, metadata and bookmarks
\usepackage[pscoord]{eso-pic} % for floating text on the page
\usepackage{calc} % for calculating lengths
\usepackage{bookmark} % for bookmarks
\usepackage{lastpage} % for getting the total number of pages
\usepackage{changepage} % for one column entries (adjustwidth environment)
\usepackage{paracol} % for two and three column entries
\usepackage{ifthen} % for conditional statements
\usepackage{needspace} % for avoiding page brake right after the section title
\usepackage{iftex} % check if engine is pdflatex, xetex or luatex

% Ensure that generate pdf is machine readable/ATS parsable:
\ifPDFTeX
    \input{glyphtounicode}
    \pdfgentounicode=1
    % \usepackage[T1]{fontenc} % this breaks sb2nov
    \usepackage[utf8]{inputenc}
    \usepackage{lmodern}
\fi



% Some settings:
\AtBeginEnvironment{adjustwidth}{\partopsep0pt} % remove space before adjustwidth environment
\pagestyle{empty} % no header or footer
\setcounter{secnumdepth}{0} % no section numbering
\setlength{\parindent}{0pt} % no indentation
\setlength{\topskip}{0pt} % no top skip
\setlength{\columnsep}{0cm} % set column seperation
\makeatletter
\let\ps@customFooterStyle\ps@plain % Copy the plain style to customFooterStyle
\patchcmd{\ps@customFooterStyle}{\thepage}{
    \color{gray}\textit{\small Kevin Klein - Página \thepage{} of \pageref*{LastPage}}
}{}{} % replace number by desired string
\makeatother
\pagestyle{customFooterStyle}

\titleformat{\section}{\needspace{4\baselineskip}\bfseries\large}{}{0pt}{}[\vspace{1pt}\titlerule]

\titlespacing{\section}{
    % left space:
    -1pt
}{
    % top space:
    0.3 cm
}{
    % bottom space:
    0.2 cm
} % section title spacing

\renewcommand\labelitemi{$\circ$} % custom bullet points
\newenvironment{highlights}{
    \begin{itemize}[
        topsep=0.10 cm,
        parsep=0.10 cm,
        partopsep=0pt,
        itemsep=0pt,
        leftmargin=0.4 cm + 10pt
    ]
}{
    \end{itemize}
} % new environment for highlights

\newenvironment{highlightsforbulletentries}{
    \begin{itemize}[
        topsep=0.10 cm,
        parsep=0.10 cm,
        partopsep=0pt,
        itemsep=0pt,
        leftmargin=10pt
    ]
}{
    \end{itemize}
} % new environment for highlights for bullet entries


\newenvironment{onecolentry}{
    \begin{adjustwidth}{
        0.2 cm + 0.00001 cm
    }{
        0.2 cm + 0.00001 cm
    }
}{
    \end{adjustwidth}
} % new environment for one column entries

\newenvironment{twocolentry}[2][]{
    \onecolentry
    \def\secondColumn{#2}
    \setcolumnwidth{\fill, 4.5 cm}
    \begin{paracol}{2}
}{
    \switchcolumn \raggedleft \secondColumn
    \end{paracol}
    \endonecolentry
} % new environment for two column entries

\newenvironment{header}{
    \setlength{\topsep}{0pt}\par\kern\topsep\centering\linespread{1.5}
}{
    \par\kern\topsep
} % new environment for the header

\newcommand{\placelastupdatedtext}{% \placetextbox{<horizontal pos>}{<vertical pos>}{<stuff>}
  \AddToShipoutPictureFG*{% Add <stuff> to current page foreground
    \put(
        \LenToUnit{\paperwidth-2 cm-0.2 cm+0.05cm},
        \LenToUnit{\paperheight-1.0 cm}
    ){\vtop{{\null}\makebox[0pt][c]{
        \small\color{gray}\textit{}\hspace{\widthof{}}
    }}}%
  }%
}%

% save the original href command in a new command:
\let\hrefWithoutArrow\href

% new command for external links:
\renewcommand{\href}[2]{\hrefWithoutArrow{#1}{\ifthenelse{\equal{#2}{}}{ }{#2 }\raisebox{.15ex}{\footnotesize \faExternalLink*}}}


\begin{document}
    \newcommand{\AND}{\unskip
        \cleaders\copy\ANDbox\hskip\wd\ANDbox
        \ignorespaces
    }
    \newsavebox\ANDbox
    \sbox\ANDbox{}

    \placelastupdatedtext
    \begin{header}
        \textbf{\fontsize{24 pt}{24 pt}\selectfont Kevin Klein}

        \vspace{0.3 cm}

        \normalsize
        \mbox{{\color{black}\footnotesize\faMapMarker*}\hspace*{0.13cm}São Paulo, SP}%
        \kern 0.25 cm%
        \AND%
        \kern 0.25 cm%
        \mbox{\hrefWithoutArrow{mailto:kevindotklein@gmail.com}{\color{black}{\footnotesize\faEnvelope[regular]}\hspace*{0.13cm}kevindotklein@gmail.com}}%
        \kern 0.25 cm%
        \AND%
        \kern 0.25 cm%
        \mbox{\hrefWithoutArrow{tel:+55-11-953115522}{\color{black}{\footnotesize\faPhone*}\hspace*{0.13cm}11 95311-5522}}%
        \kern 0.25 cm%
        \AND%
        %\kern 0.25 cm%
        %\mbox{\hrefWithoutArrow{https://yourwebsite.com/}{\color{black}{\footnotesize\faLink}\hspace*{0.13cm}yourwebsite.com}}%
        %\kern 0.25 cm%
        \AND%
        \kern 0.25 cm%
        \mbox{\hrefWithoutArrow{www.linkedin.com/in/kevin-klein-172515209}{\color{black}{\footnotesize\faLinkedinIn}\hspace*{0.13cm}Kevin Klein}}%
        \kern 0.25 cm%
        \AND%
        \kern 0.25 cm%
        \mbox{\hrefWithoutArrow{https://github.com/kevindotklein}{\color{black}{\footnotesize\faGithub}\hspace*{0.13cm}kevindotklein}}%
    \end{header}

    \vspace{0.3 cm - 0.3 cm}


    \section{Resumo}



        
        \begin{onecolentry}
            Desenvolvedor FullStack com foco em Back-end. Proeficiente em Java, Ecossistema Spring, Typescript, NodeJS, Python, SQL, NoSQL, AWS, Docker, Message Brokers, CI/CD. Com experiência comprovada
            em padrões de projeto, implementação de testes unitários, de integração, mensageria e conteinerização.
            Busco oportunidade como Desenvolvedor para contribuir com desenvolvimento de soluções escaláveis e
            de alta qualidade.
        \end{onecolentry}

        \vspace{0.2 cm}

        \begin{onecolentry}
            Sou apaixonado por programação funcional, olhe meus repositórios no \href{https://github.com/kevindotklein}{Github} se quiser saber mais sobre.
        \end{onecolentry}

        

        %\begin{onecolentry}
            %Links para projetos Open Source que contribui:  \href{https://github.com/dnl-blkv/mcdowell-cv}{Gayle McDowell}.
        %\end{onecolentry}


    
    \section{Contribuições Open Source}

    \begin{onecolentry}
        \begin{highlightsforbulletentries}


        \item \textit{CodeOdyssey}, plataforma educacional para correções automáticas de exercícios de lógica e algoritmos
        
        \href{https://github.com/ifspcodelab/codeodyssey-api}{https://github.com/ifspcodelab/codeodyssey-api}

        \vspace{0.2 cm}

        \item \textit{Bazooka}, uma biblioteca de \href{https://en.wikipedia.org/wiki/Parser_combinator}{Parser Combinators} escrita em Erlang

        \href{https://github.com/kevindotklein/bazooka}{https://github.com/kevindotklein/bazooka}
        


        \end{highlightsforbulletentries}
    \end{onecolentry}

    \section{Formação}

        
        \begin{twocolentry}{
            
        \textit{Ago 2025 – Jun 2030}}
            \textbf{IFSP - Instituto Federal de Educação, Ciência e Tecnologia de São Paulo}

            \textit{Bacharelado em Engenharia Eletrônica}
        
        \end{twocolentry}

        \vspace{0.4 cm}

        \begin{twocolentry}{
            
            
        \textit{Fev 2022 – Dez 2024}}
            \textbf{IFSP - Instituto Federal de Educação, Ciência e Tecnologia de São Paulo}

            \textit{Análise e Desenvolvimento de Sistemas}
        \end{twocolentry}

        \vspace{0.4 cm}

        \begin{twocolentry}{
            
            
        \textit{Mar 2018 – Out 2019}}
            \textbf{Kumon Ltda.}

            \textit{Inglês}

            \begin{highlights}
                \item Aluno destaque: Medalha de Ouro
            \end{highlights}
        \end{twocolentry}
        


    
    \section{Experiência}
        
        \begin{twocolentry}{ 
            
        \textit{Mai 2025 – Jun 2025}}
            \textbf{Software Developer}
            
            \textit{\href{https://www.deccoreserigrafia.com.br/}{Deccore Serigrafia}}
        \end{twocolentry}

        \vspace{0.10 cm}
        \begin{onecolentry}
            \begin{highlights}
                \item Desenvolvimento com NextJS.
                \item Suporte para mobile, internacionalização com I18N.
                \item Redução do tempo de resposta no front em 1 segundo.
                \item Uso de lambdas functions para envio de emails (removido).
            \end{highlights}
        \end{onecolentry}


        \vspace{0.2 cm}

        \begin{twocolentry}{    
            
        \textit{Abr 2023 – Dez 2023}}
            \textbf{Software Developer Intern}
            
            \textit{CodeLab}
        \end{twocolentry}

        \vspace{0.10 cm}
        \begin{onecolentry}
            \begin{highlights}
                \item Design e implementação de APIs Rest usando o ecossistema Spring e Java 21.
                \item Criação de testes unitários e de integração com Mockito, JUnit e TestContainers.
                \item Serviço de containerização usando docker escrito em Python.
                \item Comunicação assíncrona com RabbitMQ entre os Serviços.
                \item Implementei JWT com Access e Recovery Tokens com Spring Security.
            \end{highlights}
        \end{onecolentry}

    
    \section{Projetos Pessoais}



        
        \begin{twocolentry}{
            
            
        \textit{\href{https://github.com/kevindotklein/brainfuck}{brainfuck interpreter}}}
        
            \textbf{Brainfuck Interpreter}


        \end{twocolentry}

        \vspace{0.10 cm}
        \begin{onecolentry}
            \begin{highlights}
                \item Interpretador para \href{https://en.wikipedia.org/wiki/Brainfuck}{Brainfuck}.
                \item Elixir, Brainfuck.
            \end{highlights}
        \end{onecolentry}


        \vspace{0.2 cm}

        \begin{twocolentry}{
            
            
        \textit{\href{https://github.com/kevindotklein/viv}{viv}}}
            \textbf{Viv}
        \end{twocolentry}

        \vspace{0.10 cm}
        \begin{onecolentry}
            \begin{highlights}
                \item Um \textit{swap game} para mobile.
                \item React Native, Expo.
            \end{highlights}
        \end{onecolentry}


        \vspace{0.2 cm}

        \begin{twocolentry}{
            
            
        \textit{\href{https://github.com/kevindotklein/go-3d-renderer}{3D renderer}}}
            \textbf{3D Renderer}
        \end{twocolentry}

        \vspace{0.10 cm}
        \begin{onecolentry}
            \begin{highlights}
                \item Renderer 3D feito para demonstrar o funcionamento das \href{https://en.wikipedia.org/wiki/Rotation_matrix}{Rotation Matrix} convertendo coordenadas 3D em um plano 2D.
                \item Go, Ebiten
            \end{highlights}
        \end{onecolentry}



    

\end{document}